\titre{Logos} (grec) : le langage, le raisonnement\\

\titre{Définitions} : \begin{itemize}
	\item Etude du discours rationnel
	\item Etude de la raison dans le langage
	\item Science des conditions de vérité (d'une formule / d'un raisonnement)
\end{itemize}

\titre{Logique classique} : Socrate, Platon, Aristote jusqu'au début du $XIX^{eme}$ siècle. Analyse de la grammaire, prédominence de la logique aristotélicienne. \\

\titre{Logique moderne} : Symbolique et axiomatique. Mathématisée et algébrisée, elle repose sur un système hypothético-déductif et sur un système de réécriture. \\

\titre{Personnages importants} : Aristote, De Morgan, Boole, Gentzen, Russel\\

\titre{Méthodes} : L'étude des conditions de vérité d'une formule ou d'un raisonnement peut se faire par des méthodes \titre{sémantiques} ou \titre{syntaxiques}.
