\titre{Programmation impérative :} C'est un paradigme de programmation qui décrit les opérations en termes d'états du programme et de séquences d'instructions exécutées par l'ordinateur pour modifier l'état du programme. \\ Les paramètres sont passés par valeur. \\
Elle repose sur deux principes :
\begin{itemize}
	\item Séparation des programmes et des données.
	\item Priorité donnée aux traitements (alors qu'en fait ce sont les structures de données qui sont les plus stables).
\end{itemize}

On aimerait donc pouvoir manipuler des objets définis par leurs caractéristiques externes (protocoles) indépendamment de toute représentation interne $\rightarrow$ Ada (reste impératif car il permet la POO mais ne fanchit pas le pas de le rendre obligatoire). \\

\titre{Remarque :} 
\begin{itemize}
	\item Révolution Algol/Ada : associer des données à des programmes (variables locales, paramètres)
	\item Révolution Objet : associer des programmes à des données
\end{itemize}

